\input luatexja.sty
\begingroup\endlinechar=-1

\expandafter\newcount\csname [rikuri quotient]\endcsname
\expandafter\newcount\csname[rikuri remainder]\endcsname
\expandafter\gdef\csname[rikuri divide]\endcsname#1/#2{
\csname[rikuri remainder]\endcsname=#1\relax
\csname [rikuri quotient]\endcsname=\csname[rikuri remainder]\endcsname\relax
\divide\csname[rikuri quotient]\endcsname by #2\relax
\multiply\csname[rikuri quotient]\endcsname by #2\relax
\advance\csname[rikuri remainder]\endcsname by -\csname[rikuri quotient]\endcsname\relax
\divide\csname[rikuri quotient]\endcsname by #2\relax
}
%
\expandafter\newtoks\csname[rikuri cncase]\endcsname
\expandafter\newcount\csname[rikuri digit level]\endcsname
\expandafter\newtoks\csname[rikuri cnnum format]\endcsname
%
\gdef\cnnums{\csname[rikuri cnnums with case]\endcsname{num}}
\gdef\cnnumzs{\csname[rikuri cnnums with case]\endcsname{numz}}
\gdef\cnNUMs{\csname[rikuri cnnums with case]\endcsname{NUM}}
\gdef\cnNUMzs{\csname[rikuri cnnums with case]\endcsname{NUMz}}
\expandafter\gdef\csname[rikuri cnnums with case]\endcsname#1{
\begingroup % begincnnums-endcnnums search this comment to find the pair
\csname[rikuri cncase]\endcsname={#1}
\csname[rikuri cnnums]\endcsname
}
\expandafter\gdef\csname[rikuri cnnums]\endcsname{
\expandafter\futurelet
\csname[rikuri after{cnnums}]\expandafter\endcsname
\csname[rikuri{cnnums}]\endcsname
}
\expandafter\gdef\csname[rikuri{cnnums}]\endcsname{
\csname[rikuri check not digit]\endcsname
{\csname[rikuri after{cnnums}]\endcsname}
{\csname[rikuri not digit while relax]\endcsname}
%
\expandafter\ifx\csname[rikuri not digit while relax]\endcsname\relax
\endgroup % begincnnums-endcnnums search this comment to find the pair
\else
\csname[rikuri cnnums with num]\expandafter\endcsname
\fi
}
\expandafter\gdef\csname[rikuri cnnums with num]\endcsname#1{
\csname[rikuri cn\the\csname[rikuri cncase]\endcsname digit]\endcsname#1\relax
\csname[rikuri cnnums]\endcsname
}

\gdef\cnnum{\csname[rikuri cnnum with case]\endcsname{num}}
\gdef\cnnumz{\csname[rikuri cnnum with case]\endcsname{numz}}
\gdef\cnNUM{\csname[rikuri cnnum with case]\endcsname{NUM}}
\gdef\cnNUMz{\csname[rikuri cnnum with case]\endcsname{NUMz}}
\expandafter\gdef\csname[rikuri cnnum with case]\endcsname#1{
\begingroup % begincnnum-endcnnum search this comment to find the pair
\expandafter\gdef\csname[rikuri reversed num]\endcsname{}
\csname[rikuri digit level]\endcsname=0\relax
\csname[rikuri cncase]\endcsname={#1}
\csname[rikuri cnnum]\endcsname
}
\expandafter\gdef\csname[rikuri cnnum]\endcsname{
\expandafter\futurelet
\csname[rikuri after{cnnum}]\expandafter\endcsname
\csname[rikuri{cnnum}]\endcsname
}
\expandafter\gdef\csname[rikuri{cnnum}]\endcsname{
\csname[rikuri check not digit]\endcsname
{\csname[rikuri after{cnnum}]\endcsname}
{\csname[rikuri not digit while relax]\endcsname}
%
\expandafter\ifx\csname[rikuri not digit while relax]\endcsname\relax
\csname[rikuri cnnum format]\endcsname={}
\expandafter\csname[rikuri num to cnnum]\expandafter\expandafter\expandafter\endcsname
\csname[rikuri reversed num]\endcsname\relax
\endgroup % begincnnum-endcnnum search this comment to find the pair
\else
\csname[rikuri cnnum with num]\expandafter\endcsname
\fi
}
\expandafter\gdef\csname[rikuri num to cnnum]\endcsname#1{
\expandafter\ifx#10
\else
%
\csname[rikuri divide]\endcsname\the\csname[rikuri digit level]\endcsname/4\relax
\expandafter\expandafter\expandafter\ifx\csname[rikuri num to cnnum last]\endcsname 0\relax
\ifnum\csname[rikuri remainder]\endcsname=1\relax
\else
%不为1时,也就是上一个0不在个、万、亿等位。一十万不说一十零万
\csname[rikuri cnnum format]\endcsname=\expandafter{
\csname[rikuri cn\the\csname[rikuri cncase]\endcsname digit]\expandafter\endcsname\expandafter 0
\the\csname[rikuri cnnum format]\expandafter\endcsname
}
\fi\fi
%
\fi
\csname[rikuri check not digit]\endcsname
{#1}
{\csname[rikuri not digit while relax]\endcsname}
\expandafter\ifx\csname[rikuri not digit while relax]\endcsname\relax
%
\ifnum\csname[rikuri digit level]\endcsname=1\relax
\expandafter\ifx\csname[rikuri num to cnnum last]\endcsname0\relax
% 整个数值本身就是 0 时
\csname[rikuri cn\the\csname[rikuri cncase]\endcsname digit]\endcsname 0
\fi\fi
%
\the\csname[rikuri cnnum format]\endcsname
\else
\expandafter\let\csname[rikuri num to cnnum last]\endcsname#1
%
\csname[rikuri divide]\endcsname\the\csname[rikuri digit level]\endcsname/4\relax
\ifnum\csname[rikuri remainder]\endcsname=0\relax
\ifnum\csname[rikuri quotient]\endcsname>0\relax
\csname[rikuri divide]\endcsname\the\csname[rikuri quotient]\endcsname/2\relax
\csname[rikuri cnnum format]\endcsname=\expandafter{
\csname[rikuri cn\the\csname[rikuri cncase]\endcsname carryunit]\expandafter\endcsname
\the\csname[rikuri remainder]\expandafter\endcsname
\the\csname[rikuri cnnum format]\endcsname
}
\fi\fi
\ifnum#1>0\relax
\csname[rikuri divide]\endcsname\the\csname[rikuri digit level]\endcsname/4\relax
\expandafter\edef\csname[rikuri cndigit+unit]\endcsname{
\csname[rikuri cn\the\csname[rikuri cncase]\endcsname digit]\endcsname{#1}
\csname[rikuri cn\the\csname[rikuri cncase]\endcsname unit]\expandafter\endcsname
\the\csname[rikuri remainder]\endcsname
}
\csname[rikuri cnnum format]\endcsname=\expandafter\expandafter\expandafter{
\csname[rikuri cndigit+unit]\expandafter\endcsname
\the\csname[rikuri cnnum format]\endcsname
}
\fi
%
\advance\csname[rikuri digit level]\endcsname by 1\relax
\csname[rikuri num to cnnum]\expandafter\endcsname% recursive here
\fi
}
\expandafter\gdef\csname[rikuri cnnum with num]\endcsname#1{
\expandafter\edef\csname[rikuri reversed num]\endcsname{
#1
\csname[rikuri reversed num]\endcsname
}
\csname[rikuri cnnum]\endcsname
}

\expandafter\gdef\csname[rikuri check not digit]\endcsname#1#2{
% check if #1 is NOT digit
% if #1 is digit #2 will be defined to a empty macro
% if #1 is not digit #2 will be \relax
% NOTE #1 and #2 !MUST be wrapped by \csname\endcsname
\expandafter\let#2\relax
\expandafter\ifx#11\expandafter\def#2{}\else
\expandafter\ifx#12\expandafter\def#2{}\else
\expandafter\ifx#13\expandafter\def#2{}\else
\expandafter\ifx#14\expandafter\def#2{}\else
\expandafter\ifx#15\expandafter\def#2{}\else
\expandafter\ifx#16\expandafter\def#2{}\else
\expandafter\ifx#17\expandafter\def#2{}\else
\expandafter\ifx#18\expandafter\def#2{}\else
\expandafter\ifx#19\expandafter\def#2{}\else
\expandafter\ifx#10\expandafter\def#2{}\else
\fi\fi\fi\fi\fi\fi\fi\fi\fi\fi
}

\expandafter\gdef\csname[rikuri cnnumdigit]\endcsname#1{\ifcase#1{零}\or{一}\or{二}\or{三}\or{四}\or{五}\or{六}\or{七}\or{八}\or{九}\fi}
\expandafter\gdef\csname[rikuri cnnumzdigit]\endcsname#1{\ifcase#1{〇}\or{一}\or{二}\or{三}\or{四}\or{五}\or{六}\or{七}\or{八}\or{九}\fi}
\expandafter\gdef\csname[rikuri cnNUMdigit]\endcsname#1{\ifcase#1{零}\or{壹}\or{贰}\or{叁}\or{肆}\or{伍}\or{陆}\or{柒}\or{捌}\or{玖}\fi}
\expandafter\gdef\csname[rikuri cnNUMzdigit]\endcsname#1{\ifcase#1{〇}\or{壹}\or{贰}\or{叁}\or{肆}\or{伍}\or{陆}\or{柒}\or{捌}\or{玖}\fi}
\expandafter\gdef\csname[rikuri cnnumunit]\endcsname#1{\ifcase#1{}\or{十}\or{百}\or{千}\fi}
\global\expandafter\let\csname[rikuri cnnumzunit]\expandafter\endcsname\csname[rikuri cnnumunit]\endcsname
\expandafter\gdef\csname[rikuri cnNUMunit]\endcsname#1{\ifcase#1{}\or{拾}\or{佰}\or{仟}\fi}
\global\expandafter\let\csname[rikuri cnNUMzunit]\expandafter\endcsname\csname[rikuri cnNUMunit]\endcsname
\expandafter\gdef\csname[rikuri cnnumcarryunit]\endcsname#1{\ifcase#1{亿}\or{万}\fi}
\global\expandafter\let\csname[rikuri cnnumzcarryunit]\expandafter\endcsname\csname[rikuri cnnumcarryunit]\endcsname
\global\expandafter\let\csname[rikuri cnNUMcarryunit]\expandafter\endcsname\csname[rikuri cnnumcarryunit]\endcsname
\global\expandafter\let\csname[rikuri cnNUMzcarryunit]\expandafter\endcsname\csname[rikuri cnnumcarryunit]\endcsname
\endgroup%
